% preamble.tex
\usepackage[spanish]{babel}
\usepackage{geometry}
\usepackage{graphicx}
\usepackage{amssymb,amsmath,amsthm,amstext,amsfonts}
\usepackage{color}
\usepackage{listings}
\usepackage{xcolor}
\usepackage{csquotes}
\usepackage[colorlinks=true, linkcolor=linkcolor, citecolor=citecolor, urlcolor=urlcolor]{hyperref}

% Configuración de listings para código Python
\lstset{
    language=Python,
    backgroundcolor=\color{lightgray!20},
    basicstyle=\ttfamily\footnotesize,
    keywordstyle=\color{blue},
    commentstyle=\color{green!50!black},
    stringstyle=\color{red},
    numbers=left,
    numberstyle=\tiny\color{gray},
    stepnumber=1,
    breaklines=true,
    frame=single,
    captionpos=b,
}

% Entornos personalizados
\newtheorem{definition}{Definición}[section]
\newtheorem{proposition}[definition]{Proposición}
\newtheorem{remark}[definition]{Observación}
\newtheorem{notation}[definition]{Notación}
\newtheorem{lemma}[definition]{Lema}
\newtheorem{theorem}[definition]{Teorema}
\newtheorem{example}[definition]{Ejemplo}

% Otras configuraciones
\graphicspath{{Img/}}
\setlength{\parindent}{0pt}

% Configuración de biblatex
\usepackage[backend=biber, style=numeric, sorting=none]{biblatex} % Usa biber como backend
\addbibresource{bibliografia.bib} % Nombre de tu archivo .bib (sin extensión)

\usepackage{xcolor} % Carga el paquete xcolor
\definecolor{linkcolor}{RGB}{0, 0, 255}    % Define linkcolor
\definecolor{citecolor}{RGB}{0, 128, 0}    % Define citecolor
\definecolor{urlcolor}{RGB}{136, 0, 21}    % Define urlcolor

\usepackage[colorlinks=true, linkcolor=linkcolor, citecolor=citecolor, urlcolor=urlcolor]{hyperref}
