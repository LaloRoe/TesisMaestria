% capitulo2.tex
\chapter{Formulas}

\section{Integral de coseno y seno}

Vamos a resolver la siguiente integral, que nos va a ser muy útil para el desarrollo de este trabajo:

\begin{equation}\label{eq:integral_cos_sen}
    \int_{0}^{2\pi}\cos^m\theta\sin^n\theta d\theta
\end{equation}

Vamos a comenzar analizando la integral $\eqref{eq:integral_cos_sen}$ para $m,n\in\mathbb{Z}^+$, números pares.\\

Sean $n=2p$ y $m=2q$ con $p,q\in\mathbb{Z}^+$ enteros positivos.\\

\begin{equation}\label{eq:integral_cos_sen_cv}
	=\int_{0}^{2\pi}\cos^{2q}\theta\sin^{2p}\theta d\theta=\int_{0}^{2\pi}\left(\frac{1+\cos(2\theta)}{2}\right)^q\left(\frac{1-\cos(2\theta)}{2}\right)^p d\theta
\end{equation}

Hagamos un cambio de variable $\tau=\cos(2\theta)$, entonces $d\tau=-2\sin(2\theta)d\theta$ y $d\theta=-\frac{du}{2\sin(2\theta)}$. Notemos que $\tau^2=\cos^2 2\theta=1-\sin^2 2\theta$, entonces en $0\leq\theta\leq2\pi$\\

\begin{equation}\label{eq:sin_cos_2theta}
    \sin(2\theta)=\left\{\begin{array}{ll}
        \sqrt{1-\tau^2} & \text{si } \theta\in[0,\pi/2] \text{ o } \theta\in[\pi,3\pi/2]\\
        -\sqrt{1-\tau^2} & \text{si } \theta\in[\pi/2,\pi] \text{ o } \theta\in[3\pi/2,2\pi]
    \end{array}\right.
\end{equation}

Entonces la integral $\eqref{eq:integral_cos_sen_cv}$ se puede reescribir como:

\[
    =\int_{0}^{\pi/2}\left(\frac{1+\cos(2\theta)}{2}\right)^q\left(\frac{1-\cos(2\theta)}{2}\right)^p d\theta + \int_{\pi/2}^{\pi}\left(\frac{1+\cos(2\theta)}{2}\right)^q\left(\frac{1-\cos(2\theta)}{2}\right)^p d\theta
\]
\[
    + \int_{\pi}^{3\pi/2}\left(\frac{1+\cos(2\theta)}{2}\right)^q\left(\frac{1-\cos(2\theta)}{2}\right)^p d\theta + \int_{3\pi/2}^{2\pi}\left(\frac{1+\cos(2\theta)}{2}\right)^q\left(\frac{1-\cos(2\theta)}{2}\right)^p d\theta
\]

Hacemos el cambio de variable y considerando $\eqref{eq:sin_cos_2theta}$

\[
    =\int_{1}^{-1}\left(\frac{1+\tau}{2}\right)^q\left(\frac{1-\tau}{2}\right)^p\frac{-d\tau}{2\sqrt{1-\tau^2}}+\int_{-1}^{1}\left(\frac{1+\tau}{2}\right)^q\left(\frac{1-\tau}{2}\right)^p\frac{d\tau}{2\sqrt{1-\tau^2}}
\]

\[
    +\int_{1}^{-1}\left(\frac{1+\tau}{2}\right)^q\left(\frac{1-\tau}{2}\right)^p\frac{-d\tau}{2\sqrt{1-\tau^2}}+\int_{-1}^{1}\left(\frac{1+\tau}{2}\right)^q\left(\frac{1-\tau}{2}\right)^p\frac{d\tau}{2\sqrt{1-\tau^2}}
\]

\[
    =4\int_{-1}^{1}\left(\frac{1+\tau}{2}\right)^q\left(\frac{1-\tau}{2}\right)^p\frac{d\tau}{2\sqrt{1-\tau^2}} = \frac{1}{2^{q+p-1}}\int_{-1}^{1}\left(1+\tau\right)^q\left(1-\tau\right)^p\frac{d\tau}{\sqrt{1-\tau^2}}
\]

Hacemos de nuevo un cambio de variable $\tau=\cos\phi$, entonces $d\tau=-\sin\phi d\phi$. Como $-1\leq\tau\leq1$, entonces $\phi\in[0,\pi]$. En este dominio se tiene:

\[
\sqrt{1-\tau^2}=\sqrt{1-\cos^2\phi}=\sqrt{\sin^2\phi}=\left|\sin\phi\right|=\sin\phi
\]

Entonces se puede reescribir la integral como:

\[
    = \frac{1}{2^{q+p-1}}\int_{\pi}^{0}\frac{\left(1+\cos\phi\right)^q\left(1-\cos\phi\right)^p}{\sin\phi}\left(-\sin\phi\right)d\phi = \frac{1}{2^{q+p-1}}\int_{0}^{\pi}\left(1+\cos\phi\right)^q\left(1-\cos\phi\right)^p d\phi
\]

Recordemos que

\begin{equation}\label{eq: cos_phi}
    1+\cos\phi=2\cos^2\left(\frac{\phi}{2}\right)
\end{equation}

\begin{equation}\label{eq: cos_phi}
    1-\cos\phi=2\sin^2\left(\frac{\phi}{2}\right)
\end{equation}

Sustituimos las identidades \eqref{eq: cos_phi} y \eqref{eq: cos_phi} en la integral:

\[
    = \frac{1}{2^{q+p-1}}\int_{0}^{\pi}\left(2\cos^2\left(\frac{\phi}{2}\right)\right)^q\left(2\sin^2\left(\frac{\phi}{2}\right)\right)^p d\phi
\]

Desarrollando las potencias y simplificando, se tiene:

\[
=2\int_{0}^{\pi}\cos^{2q}\left(\frac{\phi}{2}\right)\sin^{2p}\left(\frac{\phi}{2}\right) d\phi
\]

Hacemos un cambio de variable $u=\frac{\phi}{2}$, entonces $du=\frac{d\phi}{2}$ y $d\phi=2du$.

Sustituimos en la integral:

\[
    =2\int_{0}^{\pi}\cos^{2q}\left(u\right)\sin^{2p}\left(u\right) 2du = 4\int_{0}^{\pi}\cos^{2q}\left(u\right)\sin^{2p}\left(u\right) du
\]

Sea $2a=2q+1$ y $2b=2p+1$, con $a,b\in\mathbb{Z}^+$ enteros positivos. Entonces la integral se puede reescribir como:

\[
    =4\int_{0}^{\pi/2}\cos^{2a-1}\left(u\right)\sin^{2b-1}\left(u\right) du = 4(\frac{1}{2}B(a,b))
\]

Donde $B(a,b)$ es la función beta.

\[
    =2B(q+\frac{1}{2},p+\frac{1}{2})=\frac{2\Gamma(q+\frac{1}{2})\Gamma(p+\frac{1}{2})}{\Gamma(q+p+1)}
\]

Donde $\Gamma(x)$ es la función gamma. Aplicando propiedades de la función gamma, se tiene:

\[
=\frac{2\frac{(2q)!}{2^{2q}q!}\sqrt{\pi}\frac{(2p)!}{2^{2p}p!}\sqrt{\pi}}{(q+p)!}=\frac{2\pi}{2^{2q+2p}}\frac{(2q)!(2p)!}{q!p!(q+p)!}
\]
Por lo tanto si $m,n$ son números enteros positivos pares, entonces:

\begin{equation}\label{eq: integral_cos_sen_pares}
    \int_{0}^{2\pi}\cos^m\theta\sin^n\theta d\theta = \frac{2\pi}{2^{m+n}}\frac{(m!)(n!)}{(\frac{m}{2})!(\frac{n}{2})!(\frac{m+n}{2})!}
\end{equation}

Ahora vamos a analizar la integral $\eqref{eq:integral_cos_sen}$ cuando al menos uno de los enteros $m$ o $n$ es impar.\\

Sea $n=2p+1$ y $m=2q$ con $p,q\in\mathbb{Z}^+$ enteros positivos.\\

\begin{equation}\label{eq:integral_cos_sen_impar_1}
    \int_{0}^{2\pi}\cos^m\theta\sin^n\theta d\theta=\int_{0}^{2\pi}\cos^{2q}\theta\sin^{2p+1}\theta d\theta
\end{equation}

Hagamos el cambio de variable $\theta=2\pi-\varphi$, sustituimos en \eqref{eq:integral_cos_sen_impar_1} y aplicamos la propiedad de linealidad de la integral:

\[
    =\int_{2\pi}^{0}\cos^{2q}(2\pi-\varphi)\sin^{2p+1}(2\pi-\varphi)(-d\varphi)=\int_{0}^{2\pi}\cos^{2q}(-\varphi)\sin^{2p+1}(-\varphi) d\varphi
\]

\[
    =\int_{0}^{2\pi}\cos^{2q}\varphi(-\sin\varphi)^{2p+1} d\varphi=-\int_{0}^{2\pi}\cos^{2q}\varphi\sin^{2p+1}\varphi d\varphi
\]

Entonces 
\[ 
    \int_{0}^{2\pi}\cos^m\theta\sin^n\theta d\theta=-\int_{0}^{2\pi}\cos^{m}\varphi\sin^{n}\varphi d\varphi
\]

Por lo tanto, para $m$ par y $n$ impar:

\begin{equation}\label{eq: integral_cos_sen_general_impar}
    \int_{0}^{2\pi}\cos^m\theta\sin^n\theta d\theta=0
\end{equation}

Análogamente podemos demostrar que, para $m$ impar y $n$ par:

\begin{equation}\label{eq: integral_cos_sen_general_impar_2}
    \int_{0}^{2\pi}\cos^m\theta\sin^n\theta d\theta=0
\end{equation}

Por lo tanto:

\begin{equation}\label{eq: integral_cos_sen_general}
    \int_{0}^{2\pi}\cos^m\theta\sin^n\theta d\theta=
\end{equation}
\begin{itemize}
    \item Si $m$ y $n$ son pares:
    \begin{equation}\label{eq: integral_cos_sen_general_pares}
        =\frac{2\pi}{2^{m+n}}\frac{(m!)(n!)}{(\frac{m}{2})!(\frac{n}{2})!(\frac{m+n}{2})!}
    \end{equation}
    \item En caso contrario:
    \begin{equation}\label{eq: integral_cos_sen_general_impar}
        =0
    \end{equation}
\end{itemize}


\section{Coordenadas polares}

Consideremos las variables $(x,y)\in\mathbb{R}^2$, las cuales están parametrizadas en términos del tiempo $t$, es decir, $x=x(t)$ y $y=y(t)$ con parámetro $t\geq0$. Definimos el cambio de coordenadas cartesianas $(x,y)$
a coordenadas polares $(r,\theta)$ mediante las relaciones:
\begin{equation}\label{eq: xpolar}
	x=r\cos(\theta)
\end{equation}

\begin{equation}\label{eq: ypolar}
	y=r\sin(\theta)
\end{equation}

donde $r=r(t)$ y $\theta=\theta(t)$ también están parametrizadas en términos de $t$.\\

Estas ecuaciones nos llevan a las siguientes identidades:

\begin{equation}\label{eq: r2}
	r^2=x^2+y^2\\
\end{equation}

\begin{equation}\label{eq: theta}
	\theta=\arctan{\left(\frac{y}{x}\right)}
\end{equation}

en este caso restringimos $\theta\in\left[-\frac{\pi}{2},\frac{\pi}{2}\right]$ ya que en ese intervalo la función $\tan\left(\theta\right)$ es invertible y continua.\\

Derivamos las ecuaciones \eqref{eq: r2} y \eqref{eq: theta} respecto a $t$, obtenemos las relaciones dinámicas en coordenadas polares:

\begin{equation}\label{eq: drcart}
	rr'=xx'+yy'
\end{equation}

\begin{equation}\label{eq: dthetacart}
	r^2\theta'=xy'-yx'
\end{equation}

Estas relaciones nos van a permitir simplificar sistemas de ecuaciones diferenciales en coordenadas cartesianas a coordenadas polares, además nos permiten estudiar la evolución radial y angular de las trayectorias en el plano de fases, lo cual es clave para analizar ciclos límite \cite{perko2001differential}.\\

