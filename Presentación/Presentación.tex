\documentclass[10pt]{beamer}
\usepackage{color}
\usepackage{graphics}
\usepackage{hyperref}
\usepackage{multimedia}
\usepackage{beamerthemeshadow}
\usepackage{epsfig}

\definecolor{tun}{rgb}{0.8,0.2,0.07}

\setbeamertemplate{background canvas}[vertical
shading][bottom=yellow!30,top=green!20]

\setbeamercovered{dynamic}

\newtheorem{teorema}{Teorema}[section]
\newtheorem{observacion}{Observaci\'on}[section]
\newtheorem{dem}{Demostraci\'on}[section]
\newtheorem{lema}{Lema}[section]
\newtheorem{proposicion}{Proposici\'on}[section]
\newtheorem{hipotesis}{Hip\'otesis}[section]
\newtheorem{definicion}{Definici\'on}[section]
\newtheorem{observation}{observation}[section]
\newtheorem{remark}{remark}[section]
\newtheorem{conclusiones}{Conclusiones}[section]
\newtheorem{final remarks}{final remarks}[section]
\newtheorem{future work collaboration}{future work collaboration}[section]
\usetheme{Darmstadt}
\usecolortheme[named=tun]{structure}
\usefonttheme{professionalfonts}
\usepackage[latin1]{inputenc}
\usefonttheme[onlylarge]{structuresmallcapsserif}
\DeclareMathOperator{\sech}{sech}

\begin{document}
\title{Teor\'ia de perturbaciones para existencia de ciclos l\'imites}

\author {\textcolor[rgb]{0.20,0.50,0.20}{ {Eduardo Ortiz Romero}}}
%\textcolor[rgb]{0.00,0.00,0.00}{}
\institute[Instituto Polit\'ecnico Nacional]{%
Comit\'e Tutorial,\\
Escuela Superior de F\'{i}sica y Matem\'{a}ticas,\\
Instituto Polit\'{e}cnico Nacional.}
\date{Posgrado. Maestr\'a en Ciencias F\'isico Matem\'aticas}

\colorlet{redshaded}{red!25!bg} \colorlet{shaded}{black!25!bg}
\colorlet{shadedshaded}{black!10!bg}
\colorlet{blackshaded}{black!40!bg}

\colorlet{darkred}{red!80!black}
\colorlet{darkblue}{blue!80!black}
\colorlet{darkgreen}{green!80!black}

\definecolor{lila}{rgb}{0.72,0.00,0.72}

\definecolor{rojo}{rgb}{0.98,0.00,0.00}

\def\radius{1.2cm}
\def\innerradius{0.85cm}

\def\softness{0.4}
\definecolor{softred}{rgb}{1,\softness,\softness}
\definecolor{softgreen}{rgb}{\softness,1,\softness}
\definecolor{softblue}{rgb}{\softness,\softness,1}

\definecolor{softrg}{rgb}{1,1,\softness}
\definecolor{softrb}{rgb}{1,\softness,1}
\definecolor{softgb}{rgb}{\softness,1,1}

\begin{frame}
\includegraphics[scale=0.08]{Img/ipn4.png}
\hfill
\includegraphics[scale=0.3]{Img/esfm4.png}
\titlepage
\end{frame}

%Slide Trayectoria académica
\section{Trayectoria acad\'emica}
\begin{frame}
    \begin{center}
        \textbf{Trayectoria acad\'emica}
    \end{center}
    \begin{scriptsize}
    \begin{table}[t]
        \begin{center}
            \begin{tabular}{|c|c|c|}
                \hline
                \multicolumn{3}{|c|}{\textbf{Cursos aprobados.}} \\ \hline
                Unidad de aprendizaje & Periodo & Calificaci\'on \\ \hline
                An\'alisis de series de tiempo	 & Agosto-Diciembre 2022 & 8 \\
                Seminario departamental I & Agosto-Diciembre 2022 & 8 \\
                Trabajo de tesis & Enero-Junio 2023 & 8 \\
                Seminario departamental II & Enero-Junio 2023 & 10 \\ 
                Ecuaciones diferenciales parciales & Enero-Junio 2023 & 8 \\
                Soluci\'on num\'erica de ecuaciones diferenciales parciales & Enero-Junio 2023 & 8 \\
                \hline
            \end{tabular}
        \end{center}    
    \end{table}
    \end{scriptsize}
    \begin{scriptsize}
        \begin{table}[t]
            \begin{center}
                \begin{tabular}{|c|c|c|}
                    \hline
                    \textbf{Cursos que estoy cursando.} \\ \hline
                    Unidad de aprendizaje \\ \hline
                    Trabajo de tesis \\
                    Ecuaciones diferenciales ordinarias \\
                    Temas selectos de ecuaciones diferenciales ordinarias \\
                    Seminario departamental II \\
                    \hline
                \end{tabular}
            \end{center}    
        \end{table}
    \end{scriptsize}
\end{frame}

%Slide 1
\section{Tesis de posgrado}
\subsection{Problema 16° de Hilbert.}
\begin{frame}



Estructura coherente electr\'on-solit\'on:
\begin{figure}
\includegraphics[scale=0.45]{electrontrans.jpg}
\caption{Transferencia electr\'onica desde un Donador (D)  a un receptor (A) a lo largo de una cadena cristalina 1D. Los resortes imitan interacciones electr\'on-fon\'on.}
\end{figure}

\end{frame}


%Slide
\begin{frame}
Hamiltoniano (aproximaci\'on de Holstein) $H=H_e+H_{ph}+H_{e-ph}$:
\begin{equation}
H_{e}=\sum_{n=-\infty}^{\infty}\varepsilon_{0}\psi_{n}^{*}\psi_{n}-J\left(\psi_{n}^{*}\psi_{n+1}+\psi_{n+1}^{*}\psi_{n}\right), \notag
\end{equation}

$\varepsilon_{0}$: Energ\'ia del electr\'on, $J$: T\'ermino de transferencia para movimiento entre sitios.
%y $q$: dipole-dipole amplitude correction factor.

\begin{equation}
H_{ph}=\sum_{n=-\infty}^{\infty}\frac{1}{2M} p_{n}^{2}\textcolor{red}{+\frac{W}{2}\left(u_{n+1}-u_{n}\right)^{2}}+V\left(u_n\right), \notag
\end{equation}

$u_n$: Desplazamiento. $p_n$: Momento. \textcolor{red}{Fon\'on longitudinal} (Fonones dispersivos de Debye o ac\'usticos). 

\begin{equation}
H_{e-ph}=\sum_{n=-\infty}^{\infty}-\chi |\psi_{n}|^2u_n.  \notag
\end{equation}

$\chi$: constante de acoplo electr\'on-fon\'on.

%\textcolor{red}{Debye's Modes}.

%\begin{observacion}
 %$\alpha=0$: Modelo de Davydov.
%\end{observacion}

\end{frame}

%Slide
\subsection{Ecuaciones de movimiento}
\begin{frame}
Ecuaciones de movimiento:
\begin{eqnarray}
i\hbar \frac{d\psi_{n}}{dt}&=&-J\left(\psi_{n-1}+\psi_{n+1}\right)-\chi u_n\psi_{n}, \notag \\
M \frac{d^2{u}_{n}}{dt^2}&=&\textcolor{red}{W\left(u_{n-1}-2u_{n}+u_{n+1}\right)}-V'\left(u_n\right)+\chi \left|\psi_{n}\right|^{2}.  \notag
\end{eqnarray}

\begin{figure}
\includegraphics[width=2.5 in,height=2 in]{MorsePot.eps}
\caption{Potenciales $V(r)=D\left(e^{-\alpha r}-1\right)^2$ y $V_{ap}=D\alpha^2r^2\left(1-\alpha r\right)$}
\end{figure}

\end{frame}

%Slide
\section{L\'imite continuo}

\begin{frame}



\end{frame}

\section{Conclusiones}
\begin{frame}

\begin{conclusiones}
\begin{enumerate}
\item Propagaci\'on 
\end{enumerate}
\end{conclusiones}

\end{frame}


\end{document}
