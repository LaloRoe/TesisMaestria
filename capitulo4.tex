\chapter{Sistemas Liénard}

Los sistemas Liénard representan una clase importante de ecuaciones diferenciales no lineales que han sido fundamentales en el estudio de oscilaciones no lineales y ciclos límite. Estos sistemas fueron introducidos por Alfred-Marie Liénard en 1928 como un modelo para describir fenómenos oscilatorios en sistemas físicos, particularmente en circuitos eléctricos.

\section{Definición y Formulación General}

Un sistema Liénard es una ecuación diferencial de segundo orden de la forma:
\begin{equation}\label{eq:lienard}
	x'' + f(x)x' + g(x) = 0
\end{equation}
donde $f(x)$ y $g(x)$ son funciones continuas definidas en $\mathbb{R}$. Para analizar esta ecuación como un sistema dinámico, se realiza una transformación a un sistema de primer orden. Definimos:
\[
F(x) = \int_0^x f(s)  ds
\]
y realizamos el cambio de variable $y = x' + F(x)$. Con este cambio, obtenemos el sistema equivalente:
\begin{equation}\label{eq:lienardsystem}
	\begin{aligned}
		x' &= y - F(x), \\
		y' &= -g(x).
	\end{aligned}
\end{equation}

Este sistema tiene una estructura interesante que permite estudiar sus propiedades cualitativas, particularmente en relación con la existencia y unicidad de ciclos límite.

\section{Teorema de Liénard}

El teorema de Liénard proporciona condiciones suficientes para garantizar la existencia de un único ciclo límite estable en el sistema $\eqref{eq:lienardsystem}$. Este resultado es fundamental en la teoría de ciclos límite y ha sido ampliamente utilizado en aplicaciones prácticas.

\begin{theorem}[Liénard]
	Si las funciones $F(x)$ y $g(x)$ del sistema $\eqref{eq:lienardsystem}$ satisfacen las siguientes hipótesis:
	\begin{enumerate}
		\item $F, g \in C^1(\mathbb{R})$, es decir, son continuamente diferenciables.
		\item $F$ y $g$ son funciones impares, es decir, $F(-x) = -F(x)$ y $g(-x) = -g(x)$.
		\item $x g(x) > 0$ para todo $x \neq 0$. Esto implica que $g(x)$ tiene el mismo signo que $x$, asegurando que la fuerza restauradora sea dirigida hacia el origen.
		\item $F'(0) < 0$, lo que indica que cerca del origen la fricción es positiva (disipativa).
		\item $F(x)$ tiene ceros solo en $x = 0$ o en $x = \pm a$, donde $a > 0$. Esto asegura que la función $F(x)$ tenga un comportamiento específico que permita la formación de ciclos límite.
		\item $F(x)$ es monótona creciente para $x > a$ y satisface $\lim_{x \to \infty} F(x) = \infty$. Esta condición garantiza que el sistema tenga un comportamiento asintótico controlado.
	\end{enumerate}
	entonces el sistema $\eqref{eq:lienardsystem}$ tiene un único ciclo límite estable.
\end{theorem}

\subsection{Interpretación del Teorema de Liénard}

El teorema de Liénard combina condiciones sobre la fricción no lineal $F(x)$ y la fuerza restauradora $g(x)$ para garantizar la existencia de un ciclo límite estable. La condición de que $F(x)$ sea impar y tenga ceros específicos refleja un balance entre disipación de energía y acumulación de energía en el sistema. Además, la monotonía de $F(x)$ para valores grandes de $x$ asegura que las trayectorias no diverjan indefinidamente.

\section{Aplicación: Oscilador de Van der Pol}

El oscilador de Van der Pol es uno de los ejemplos más conocidos de un sistema Liénard. Su ecuación diferencial está dada por:
\[
x'' + \epsilon (x^2 - 1)x' + x = 0,
\]
donde $\epsilon > 0$ es un parámetro que controla la no linealidad del sistema. Este modelo fue originalmente desarrollado para describir oscilaciones en circuitos eléctricos con válvulas triodo.

Transformación al Sistema Liénard

Definimos:
\[
F(x) = \int_0^x \epsilon (s^2 - 1) \, ds = \frac{\epsilon}{3} (x^3 - 3x),
\]
y $g(x) = x$. El sistema equivalente es:
\[
\begin{aligned}
	x' &= y - \frac{\epsilon}{3} (x^3 - 3x), \\
	y' &= -x.
\end{aligned}
\]

Verificación de las Hipótesis del Teorema de Liénard

Verifiquemos que $F(x)$ y $g(x)$ satisfacen las hipótesis del teorema de Liénard:

1. **Continuidad y Diferenciabilidad**: Ambas funciones $F(x)$ y $g(x)$ son continuamente diferenciables ($C^1$).
2. **Imparidad**: 
   \[
   F(-x) = \frac{\epsilon}{3} [(-x)^3 - 3(-x)] = -\frac{\epsilon}{3} (x^3 - 3x) = -F(x),
   \]
   y
   \[
   g(-x) = -x = -g(x).
   \]
3. **Signo de $x g(x)$**: Para $x \neq 0$, $x g(x) = x^2 > 0$.
4. **Condición en el origen**: $F'(x) = \epsilon (x^2 - 1)$, por lo que $F'(0) = -\epsilon < 0$.
5. **Ceros de $F(x)$**: Resolviendo $F(x) = 0$, obtenemos $x = 0$ y $x = \pm \sqrt{3}$.
6. **Monotonía y límite**: Para $x > \sqrt{3}$, $F'(x) > 0$, y $\lim_{x \to \infty} F(x) = \infty$.

Por lo tanto, todas las hipótesis del teorema de Liénard se cumplen, y concluimos que el oscilador de Van der Pol tiene un único ciclo límite estable.

\section{Resultados Relevantes y Extensiones}

El teorema de Liénard ha inspirado numerosos resultados y extensiones en la teoría de ciclos límite. Algunos de ellos incluyen:

1. **Generalización a Sistemas Perturbados**: Se han estudiado versiones perturbadas de sistemas Liénard para analizar cómo pequeñas perturbaciones afectan la existencia y estabilidad de ciclos límite.
2. **Relación con el Teorema de Poincaré-Bendixson**: El teorema de Liénard puede verse como una aplicación específica del teorema de Poincaré-Bendixson en dos dimensiones, que garantiza la existencia de conjuntos límite compactos.
3. **Estabilidad Global**: Bajo ciertas condiciones adicionales, se puede demostrar que el ciclo límite en un sistema Liénard es globalmente atractivo.

\section{Ejemplo Adicional: Oscilador de Rayleigh}

Otro ejemplo clásico de un sistema Liénard es el oscilador de Rayleigh, dado por:
\[
x'' + \epsilon \left( \frac{1}{3} (x')^3 - x' \right) + x = 0.
\]
Este sistema también puede transformarse en un sistema Liénard y verificar las hipótesis del teorema correspondiente.

\section{Conclusión}

Los sistemas Liénard constituyen una herramienta poderosa para el análisis de oscilaciones no lineales y ciclos límite. Su estudio no solo proporciona resultados teóricos rigurosos, sino que también tiene aplicaciones prácticas en diversos campos, como la ingeniería eléctrica, la biología y la física. El teorema de Liénard es un resultado central que garantiza la existencia y unicidad de ciclos límite bajo condiciones específicas, y su aplicación a modelos como el oscilador de Van der Pol ilustra su relevancia en problemas reales.
