
\chapter{Generalización}
%%%%%%%%%%%%%%%%%% Lema Propiedades del conjunto w-límite %%%%%%%%%%%%%%%%%%%%%%%%%%
\begin{lemma}
	\textbf{Propiedades del conjunto $\omega$-límite.}\\
	Sea $\vec{x}\in\mathbb{R}^n$ tal que $\varGamma_{\vec{x}}^{+}$ esta acotada. Entonces:
	\begin{enumerate}
		\item $\omega(\vec{x})\neq\emptyset$
		\item $\omega(\vec{x})$ es acotado.
		\item $\omega(\vec{x})$ es cerrado.
		\item $\omega(\vec{x})$ es conexo.
	\end{enumerate}
\end{lemma}

%%%%%%%%%%%%%%%%% Convergencia %%%%%%%%%%%%%%%%%%%%%%%%%%%%%%%%%%%%%
\begin{lemma}
	\textbf{Convergencia.}\\
	Sea $\vec{x}\in\mathbb{R}^n$ con $\varGamma_{\vec{x}}^{+}$ acotada. Entonces $\varphi(\varphi(t,\vec{x}),\omega(\vec{x}))\to_{t\to\infty}0$
\end{lemma}

%%%%%%%%%%%%%%%%% Invarianza %%%%%%%%%%%%%%%%%%%%%%%%%%%%%%%%%%%%%%
\begin{lemma}
	\textbf{Invarianza.}\\
	Sea $\varGamma_{\vec{x}}^{+}$ acotada. Entonces $\omega(\vec{x})$ es invariante bajo el flujo $\varphi^+$, es decir, para cada $\vec{y}\in\omega(\vec{y})$
	entonces $\vec{z}=\varphi^+(\vec{x})\in\omega(\vec{x})$ \\ ($\varphi^t(\omega(\vec{x}))\subset\omega(\vec{x})$).
\end{lemma}

%%%%%%%%%%%%%%%% Transitividad %%%%%%%%%%%%%%%%%%%%%%%%%%%%%%%%%%%
\begin{lemma}
	Transitividad.\\
	Sean $\vec{x},\vec{y},\vec{z}\in\mathbb{R}^n$. Si $\vec{z}\in\omega(\vec{y})$ y $\vec{y}\in\omega(\vec{x})$.
	\\ Entonces: $\vec{z}\in\omega(\vec{x})$.
\end{lemma}

Ahora consideremos sólo sistemas planares:
\begin{eqnarray}
	x'=f(x,y) \\
	y'=g(x,y)
\end{eqnarray}
Esperamos que los conjuntos $\omega$ y $\alpha$-límites son equilibrios de la EDO o cíclos(órbitas periódicas).

%%%%%%%%%%%%%%%%%%%%%%% Segmento Transverso %%%%%%%%%%%%%%%%%%%%%%%%%%
\begin{definition}\\
	Decimos que el segmento cerrado $L\subset\mathbb{R}^2$ es \textbf{transverso} al campo
	$$\vec{F}(x,y)=(f(x,y),g(x,y))$$
	definido por \eqref{eq1}, si $\vec{F}$ ni se anula ni es tangente en $L$.
\end{definition}

%%%%%%%%%%%%%%%%%%%%%% Curva de Jordan %%%%%%%%%%%%%%%%%%%%%%%%%%%%%%
\begin{definition}
	La imagen continua de un circulo define a una curva de Jordan, es decir, una curva simple rectificable.
\end{definition}

%%%%%%%%%%%%%%%%%%%%%% Teorema  Curva de Jordan %%%%%%%%%%%%%%%%%%%%%%%%%%%%%%
\begin{theorem} \textbf{Curva de Jordan}\\
	Sea $\varGamma\subseteq\mathbb{R}^2$ curva de Jordan. Entonces $\varGamma^C=Ext(\varGamma)\cup Int(\varGamma)$ con
	$Ext(\varGamma)\cap Int(\varGamma)=\emptyset$.
	\\ $Ext(\varGamma)$, $Int(\varGamma)\equiv$ Conexos, abiertos y $Ext(\varGamma)\equiv$ No acotado y
	$Int(\varGamma)\equiv$ Acotado.
\end{theorem}

%%%%%%%%%%%%%%%%%%%%%% Monotonicidad %%%%%%%%%%%%%%%%%%%%%%%%%%%%%%%%%%%%%%%%%%%
\begin{lemma}
	\textbf{Monotonicidad}\\
	Si $\varGamma$ órbita solución de interseca a un segundo transverso $L$ para $t_i\to\infty$.\\
	Entonces, la sucesión de intersecciones $\{\vec{p_i}\}_{i=1}^\infty$ o es constante o en monótona
\end{lemma}

%%%%%%%%%%%%% Lema %%%%%%%%%%%%%%%%%%%%%%%%%%%%%%%%%%%
\begin{lemma}
	\\ Sean $\vec{p}\omega(\vec{x})$ y $L$ transverso por $\vec{p}$. Entonces existe $\{t_i\_{i=1}^\infty}$ sucesión creciente, $t_i\to\infty$ tal que
	$$\vec{q_i}=\varphi(t_i,\vec{x})\to\vec{p}$$
	con $\vec{q_i}\in L$.
\end{lemma}

%%%%%%%%%%%%%%%%%%%%%%% Nota %%%%%%%%%%%%%%%%%%%%%%%%%%%%%%
\textbf{Nota}\\
Resta probar que $ $

%%%%%%%%%%%%%%%%%%%% Lema %%%%%%%%%%%%%%%%%%%%%%%%%%%%%%%
\begin{lemma}\\
	El conjunto $\omega(\vec{x})$ puede intersectar al segmento transverso $L$ en a lo más un punto
\end{lemma}

%%%%%%%%%%%%%%%%%%% Lema %%%%%%%%%%%%%%%%%%%%%%%%%%%%%%%
\begin{lemma}\\
	Si $\omega(x)\neq\emptyset$ y no contiene puntos de equilibrio. Entonces contiene a una órbita periódica $\varGamma_0$.
\end{lemma}

%%%%%%%%%%%%%%%%%%%% Lema %%%%%%%%%%%%%%%%%%%%%%%%
\begin{lemma}\\
	Si $\omega(\vec{x})$ contiene una órbita periódica $\varGamma_0$. Entonces $\omega(\vec{x})=\varGamma_0$.
\end{lemma}

%%%%%%%%%%%%%%%%%%%% Teorema Poincaré - Bendixson %%%%%%%%%%%%%%%%%%%%%%%%%%%%
\begin{theorem}[Poincaré - Bendixson]
	Sea $\vec{x}\in\mathbb{R}^2$ tal que $\varGamma_{\vec{x}}^+\subseteqD\subseteq\mathbb{R}^2$ con $D\equiv$ Cerrado y
	acotado y conteniendo un número finito de equilibrios de la EDO:\\
	\begin{eqnarray}4
		x'=f(x,y) \\
		y'=g(x,y).
	\end{eqnarray}
	Entonces se cumple algunas de las siguientes
	\begin{enumerate}
		\item $\omega(\vec{x})=\omega(\varGamma_{\vec{x}}^+)$ esta formado por un equilibrio.
		\item $\omega(\vec{x})$ es UNA órbita periódica.
		\item $\omega(\vec{x})$ está formada por equilibrios y órbitas que tienen a dichos equilibrios
		      y órbitas que tienen a dichos equilibrios como puntos $\alpha$ o $\omega$-límite.
	\end{enumerate}
\end{theorem}

%%%%%%%%%%%%%%%%%%%%%%%%%%%%% Nota %%%%%%%%%%%%%%%%%%%%%%%%%%%%%%%%%
\textbf{Nota}\\
\begin{enumerate}
	\item Si $\vec{F}(x,y)=(f(x,y),g(x,y))$ es analítico en $(x,y)$ donde $x'=f(x,y)$ , $y'=g(x,y)$.
	      Entonces $\vec{F}$ tiene un número finito de equilibrios en todo compacto de $\mathbb{R}^2$.
	      Por ejemplo, $f,g\equiv$ poinomiales en $x$ y $y$.
	\item Por el lema de Convergencia
	      $$\rho(\varphi(t,\vec{x}),\omega(\vec{x}))\to_{t\to\infty}0$$
	      y iii) se implica que $\omega(\vec{x})\equiv$ Consistente de un contorno formado por equilibrios y sus órbitas conectoras llamadas
	      \textbf{heteroclínicas}/\textbf{homoclínicas}.
\end{enumerate}

%%%%%%%%%%%%%%%%%%%%%%%% Definición %%%%%%%%%%%%%%%%%%%%%%%%%%%%%%%
\begin{definition}
	\textbf{Curva homoclínica}\\
	La órbita que conecta a un punto silla consigo mismo (es decir, la intersección de la variedad estable con la inestable)
	se le llama curva homoclínica.
\end{definition}

%%%%%%%%%%%%%%%%%%%%%%%%% Definición %%%%%%%%%%%%%%%%%%%%%%%%%%%%%%
\begin{definition}
	La órbita que conecta dos diferentes puntos de equilibrio se le llama curva heteroclinica.
\end{definition}

%%%%%%%%%%%%%%%%%%%%%%%% Teorema %%%%%%%%%%%%%%%%%%%%%%%%%%%%%%%%%
\begin{theorem}
	Sean $C_1,C_2\subseteqD\subseteq\mathbb{R}^2$ compacto y $C_1,C_2\equiv$ Curvas simples cerradas tales que
	$\vec{F}(x,y)=(f(x,y),g(x,y))|_{C_1,C_2}$ entra/sale de $C_1/C_2$. Entonces existe ciclo límite entre $C_1$ y $C_2$.
\end{theorem}

%%%%%%%%%%%%%%%%%%%% Nota $$$$$$$$$$$$$$$$$$$$$$$$$$$$$$$$
\textbf{Nota}\\
$C_1,C_2$. No necesariamente son órbitas solución.

%%%%%%%%%%%%%%%%%% Definición %%%%%%%%%%%%%%%%%%%%%%%%%%%%%%%%%
\begin{theorem}[Criterio de Bendixson]
	Sea $\vec{x'}=f(x,y)$, $\vec{y'}=g(x,y)$ con $\vec{F}=(f,g):\Omega\subseteq\mathbb{R}^2\to\mathbb{R}^2$, $\vec{F}\in C'(\Omega)$,
	$\Omega\equiv$ Simple conectado tales que
	$$div\vec{F}=\vec{\nabla}\vec{F}=\frac{\partial f}{\partial x}+\frac{\partial g}{\partial y}.$$
	No es identicamente cero en cada abierto de $\Omega$ y no cambia de signo.\\
	Entonces no existe órbita periódica contenida en $\Omega$.
\end{theorem}

%%%%%%%%%%%%%%%%%%%%%%%%% Teorema %%%%%%%%%%%%%%%%%%%%%%%
\begin{theorem}[Criterio de Dulac]
	Lo mismo que Bendixson, pero
	$$\vec{F}\equiv(\mu(x,y)\vec{F}(x,y))$$
	con:
	$$\mu(x,y)\equiv\text{Campo Escalar positivo}.$$
\end{theorem}

%%%%%%%%%%%%%%%%%%%%%%% Teorema %%%%%%%%%%%%%%%%%%%%%%%%%%%
\begin{theorem}
	El sistema gradiente:
	$$\vec{x'}=-\nabla V(\vec{x})$$
	No tiene órbitas periódicas.
\end{theorem}

\chapter{Aplicaciones: Modelos lotka-Volterra}

\begin{enumerate}
	\item Competición de Especies.\\
	      Sea
	      $$v\equiv\text{densidad de presas}$$
	      $$p\equiv\text{densidad de depredadores}$$
	      En ausencia de depredador $v\equiv$ crece exponencialmente, es decir:
	      $$v'=av\text{, }\text{ }a\equiv\text{Tasa de crecimiento}.$$
	      En ausencia de presas $p\equiv$ decrece esponencialmente, es decir:
	      $$p'=-cp\text{, }\text{ }c\equiv\text{Tasa de decrecimiento}$$
	      En la interacción, por la ley de acción de masas (es decir, variación es proporcional al número de contactos).
	      $\approx b\vee p$, $b\equiv$constante de proporcionalidad para presas y $d\equiv$constante de proporcionalidad
	      para depredadores, por lo tanto el sistema es:
	      $$v'=av-b\vee p=f(v,p)$$
	      $$p'=-cp+d\vee p=g(v,p)$$
	      $$:\mathbb{R}^2\to\mathbb{R}^2$$
	      $$(v,p)\to(f(v,p),g(v,p))$$
	      Nullclinas:
	      $$v'=0\text{, es decir }v(a-bp)=0$$
	      $$p'=0\text{, es decir }p(-c+dv)=0.$$
	      Entonces:
	      $$v=0\text{ o }p=\frac{a}{b}$$
	      $$p=0\text{ o }v=\frac{c}{d}$$
	      Linearización:\\
	      En $\vec{x^*}=(0,0)$, para $v\ll 1$, $p\ll 1$\\
	      \[
		      \begin{matrix}
			      v'\approx av \\
			      p'\approx -cp
		      \end{matrix}
		      \Longrightarrow
		      \begin{pmatrix}
			      a & 0  \\
			      0 & -c
		      \end{pmatrix}
		      \begin{pmatrix}
			      v \\
			      p
		      \end{pmatrix}
		      =
		      \begin{pmatrix}
			      v' \\
			      p'
		      \end{pmatrix}
	      \]
	      Así, $\lambda_1=a>0$ y $\lambda_2=-c<0$
	      $$\therefore\vec{x^*}=(0,0)\text{: Silla}$$
	      Luego
	      \[
		      \begin{matrix}
			      (v,p) \\
			      (v,0) \\
			      (0,p)
		      \end{matrix}
		      \Longrightarrow
		      \begin{matrix}
			      (f(v,p),g(v,p)) \\
			      (av,0)          \\
			      (0, -cp).
		      \end{matrix}
	      \]
	      En $\vec{x^*}=(\frac{c}{d},\frac{a}{b})$
	      \[
		      J=
		      \begin{pmatrix}
			      0            & -\frac{bc}{d} \\
			      \frac{ad}{b} & 0
		      \end{pmatrix}
		      ;
		      \begin{matrix}
			      p(\lambda)=\lambda^2+ac=0 \\
			      \lambda_{1,2}=\pm i\sqrt{ac}
		      \end{matrix}
	      \]
	      por lo tanto \vec{x*} centro.\\
	      \\ Por otro lado, notemos que:
	      $$\frac{dv}{dp}=\frac{v(a-bp)}{p(-c+dv)}$$
	      Es decir:
	      $$\displaystyle\int\frac{-c+dv}{v}dv=\int\frac{a-bp}{p}dp$$
	      Luego:
	      $$-c\ln(v)+dv=a\ln(p)-bp+\text{cte}.$$
	      Es decir:
	      $$L=L(v,p)=-c\ln(v)-a\ln(p)+dv+bp=\text{cte}.$$
	      Luego:
	      $$\frac{dL}{dt}=0$$
	      $L$ cantidad conservada.\\
	      \\ Se puede ver que si $V=\frac{c}{d}+\tilde{V}$, $\tilde{V}\ll 1$ y $p=\frac{a}{b}+\tilde{p}$, $\tilde{p}\ll 1$.
	      Taylor:
	      $$\ln(V)=\ln(\frac{c}{d}+\tilde{V})=\ln{\frac{c}{d}}+\tilde{V}\frac{d}{c}+O(\tilde{V})$$
	      $$\ln(p)=\ln(\frac{a}{b}+\tilde{p})=\ln(\frac{a}{b})+\tilde{p}\frac{b}{a}-\frac{\tilde{p}^2}{2}\frac{b^2}{a^2}+O(\tilde{p}^2)$$
	      Así, en la vecindad de $(\frac{c}{d},\frac{a}{b})$.
	      $$\text{cte}=L(v,p)=-c\ln(\frac{c}{d})-d\tilde{v}+\frac{d^2}{2c}\tilde{v}^2$$
	      $$=-a\ln(\frac{a}{b})-b\tilde{p}+\frac{b^2}{2a}\tilde{p}^2+dv+bp.$$
	      Por lo tanto:
	      $$\frac{d^2}{2c}\tilde{v}^2+\frac{b^2}{2a}\tilde{p}^2=\text{cte'}>0.$$

	\item Modelo de la Glucósis.

	      $$u'=-u+av+u^2v=f(u,v)$$
	      $$v'=b-av-u^2v=g(u,v)$$
	      $a,b>0$, $u\equiv ADP$ y $v\equiv ATP$.

	      Nullclinas:
	      $$0=f(u,v)=-u+v(a+u^2)$$
	      $$0=g(u,v)=b-av-u^2v.$$
	      Es decir:
	      \[
		      v=\frac{u}{a+u^2} \text{ y }
		      v=\frac{b}{a+u^2}
	      \]
	      Único equilibrio:
	      $$\frac{u}{a+u^2}=\frac{b}{a+u^2}$$
	      es decir, $u=b$ y entonces:
	      $$v=\frac{b}{a+b^2}$$
	      Linearización en el equilibrio:
	      \[
		      J=
		      \begin{pmatrix}
			      -1+2uv & a+u^2  \\
			      -2uv   & -a-u^2
		      \end{pmatrix}_{\vec{x^*}}
		      =
		      \begin{pmatrix}
			      -1+\frac{2b^2}{a+b^2} & a+b^2  \\
			      -\frac{2b^2}{a+b^2}   & -a-b^2
		      \end{pmatrix}
		      = A.
	      \]
	      Por lo tanto:
	      $$0=p(\lambda)=\lambda^2-tr(A)\lambda+|A|$$
	      con:
	      $$tr(A)=\frac{-a+b^2}{a+b^2}-a-b^2$$
	      y
	      $$|A|=a-b^2+2b^2=a+b^2>0.$$
	      $$\therefore \lambda_{1,2}=\frac{tr(A)\pm\sqrt{(tr(A))^2-4|A|}}{2}.$$
	      Tenemos equilibrio
	      \[
		      \text{Inestable}
		      \left\{ \begin{array}{lcc}
			      \text{Espiral} \\
			      \text{Nodo}
		      \end{array}
		      \right.
		      \text{ si }
		      \text{ }
		      tr(A)>0
	      \]
	      En $tr(A)=0$, entonces:
	      $$-a+b^2=(a+b^2)^2$$
	      $$-a+b^2=a^2+b^4+2ab^2$$
	      $$\therefore b^2=\frac{1-2a\pm\sqrt{(2a-1)^2-4(a^2+a)}}{2}$$
\end{enumerate}
